\documentclass{beamer}
% Преамбула
\title{My Super Topic}
\author{Ashot and Alice}
\usepackage[utf8]{inputenc}
\usepackage[T2A]{fontenc}     
\usepackage[english,russian]{babel} 
\usepackage{amsfonts}
\usepackage{amsmath}
\usepackage{graphicx}
\graphicspath{ {./images/} }

\begin{document}

\begin{frame}
\thispagestyle{empty}
\centerline{Московский государственный университет}
\vfill
\centerline{Факультет вычислительной математики и кибернетики}
\vfill
\vfill
\vfill
\vfill
\Large
\begin{centering}
{\bfseries "Численное решение краевой задачи Дирихле для уравнения Лапласа"}
\end{centering}
\normalsize
\vfill
\vfill
\vfill
\begin{flushright}
Студент: Бурцев Леонид\\
Группа: 303
\end{flushright}
\vfill
\vfill
\centerline{Москва}
\centerline{2023}
\pagebreak
\end{frame}

\begin{frame}
\textbf{Описание исходной задачи}
\newline
Необходимо решить краевую задачу Дирихле для уравнения Лапласа

\begin{equation*}
 \begin{cases}
   -u'' = f\\
   u(0) = a, u(1) = b 
 \end{cases}
\end{equation*}
численно с помощью метода конечных разностей.
\newline
Решаем задачу на интервале $(0, 1)$, вводя на ней равномерную сетку $x_0, x_1, ..., x_N$,
$x_i = i * h$,
$h = \frac{1}{N}$
Дискретная аппроксимация уравнения:
$$-\frac{y_{i-1}-2y_i+y_{i+1}}{h^{2}} = f(x_i)$$

для приграничных узлов $(x_1, x_{N-1})$
сюда войдут граничные условия
\end{frame}

\begin{frame}
\textbf{Метод прогонки: }
\newline
Определим $a_i$ как элементы стоящие на поддиагонали в $i$-ой строке
\newline
Определим $b_i$ как элементы стоящие на диагонали в $i$-ой строке
\newline
Определим $c_i$ как элементы стоящие на наддиагонали в $i$-ой строке
\newline
Определим $d_i$ как элемент правой части в $i$-ой строке
\newline

Прямая прогонка состоит в вычислении прогоночных коэффициентов $\alpha_i$ и $\beta_i$ , где $i$ – номер строки матрицы. Этот этап выполняется при $i = 1...n$ строго по возрастанию значения $i$.

1)В первой строке матрицы $i = 1$ используются формулы:
\[\mathbf{y_1=b_1, \alpha_1=\frac{-c_1}{y_1}, \beta_1=\frac{d_1}{y_1}}\]
\end{frame}

\begin{frame}
2) Для строк $i$ от $2$ до $N-2$ используются рекуррентные формулы:
\[\mathbf{y_i=b_i+a_i*\alpha_{i-1}, \alpha_i=\frac{-c_i}{y_i}, \beta_i=\frac{(d_i-a_i*\beta_{i-1})}{y_i}}\]
3) При $i = N - 1$ прямая прогонка завершается вычислением: 
\[\mathbf{y_{N-1}=b_{N-1}+a_{N-1}*\alpha_{N-2}, \beta_{N-1}=\frac{(d_{N-1}-a_{N-1}*\beta_{N-2})}{y_{N-1}}}\]

После этого производится обратная прогонка, в которой происходит вычисление неизвестных $yi$. Этот этап выполняется при $i = n...1$ строго по
убыванию значения $i$.

4) В последней строке матрицы $i = N - 1$ выполнено $x_{N-1} = \beta_{N-1}$
\newline
5) Для всех остальных строк при $i$ от $N-2$ до $1$ применяется формула:
\[\mathbf{x_i=\alpha_i*x_{i+1}+\beta_{i}}\]
\end{frame}

\begin{frame}
\textbf{Для функции $u=sin(\pi(x))$ графики зависимости точности с-нормы и дискретной l2-нормы от шага сетки в логарифмическом виде выглядят следующим образом}

\begin{figure}
    \centering
    \includegraphics[width=0.99\linewidth]{Снимок экрана 2024-02-26 в 12.56.43.png}
    \caption{Enter Caption}
    \label{fig:enter-label}
\end{figure}

\end{frame}

\begin{frame}
\textbf{Сходимость l2-нормы: $O(0.82 * h^{2.5})$}    
\newline
\textbf{Сходимость c-нормы: $O(0.91* h^{2})$}    
\end{frame}
\end{document}
